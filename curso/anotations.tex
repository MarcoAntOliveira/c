\documentclass{article}
\usepackage[a4paper, left=0.5cm, right=0.5cm, top=0.5cm, bottom=0.5cm]{geometry}
\usepackage{listings}
\date{}
\usepackage{graphicx} 
\usepackage{color}
\usepackage{multicol} % Adicionado o pacote multicols
\lstset{
  language=c,
  basicstyle=\ttfamily,
  keywordstyle=\bfseries\color{red},
  commentstyle=\color{blue},
  stringstyle=\color{red!70!black},
  numberstyle=\tiny,
  stepnumber=1,
  numbersep=5pt,
  backgroundcolor=\color{white},
  breaklines=true,
  breakautoindent=true,
  showspaces=false,
  showstringspaces=false,
  showtabs=false,
  tabsize=2
}

\title{\textbf{\textcolor{blue}{Language C anotations}}}
\begin{document}
\maketitle
\begin{multicols}{2}
\textcolor{blue}{\section{\textcolor{blue}{Introduction}}}
\textcolor{red}{\subsection{\textbf{\textcolor{red}{Cast}}}}
\begin{lstlisting}
    int a =3;
    int b = 2;
    double pontos = a /(double) b ;// cast de int para double
    float pi = 3.1415;
    int pi_convert =(int)pi;  
\end{lstlisting}
\textcolor{red}{\subsection{Booleano em c}}
Os testes de booleano em c acontecem por meio de um numero inteiro, ou seja ele representa true por 1 e false por 0. 
\begin{lstlisting}
    int acertou = (escolha_jogador== numero_secreto);
    int maior =  (escolha_jogador >numero_secreto);
\end{lstlisting}
O codigo acima faz teste lógico para as duas variaveis no codigo em questão, eram usando em condicionais , o que tornava o código mais legível.
\textcolor{red}{\subsection{Obtendo numeros randomicos}}
O trecho de código obtem numero randomicos, a cada segundo devido p uso da função time. 

\begin{lstlisting}
  int segundos =  time(0);
  srand(segundos);
  int numero_grande = rand()%100;
\end{lstlisting}




\end{multicols}



\end{document}

