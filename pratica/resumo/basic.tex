\documentclass{article}
\usepackage[legalpaper, left=1 cm, right=0.5cm, top=1cm, bottom=0.5cm] {geometry}
\date{} % Remove a exibição da data
\usepackage{xcolor}
\usepackage{listings}
\usepackage{graphicx}
\usepackage[utf8]{inputenc}
\usepackage[T1]{fontenc}

\lstset{
  language=c,
  basicstyle=\ttfamily,
  keywordstyle=\bfseries\color{blue},
  commentstyle=\color{blue},
  stringstyle=\color{red!70!black},
  numberstyle=\tiny,
  stepnumber=1,
  numbersep=5pt,
  backgroundcolor=\color{white},
  breaklines=true,
  breakautoindent=true,
  showspaces=false,
  showstringspaces=false,
  showtabs=false,
  tabsize=2
}
\title{Modulo1}
\begin{document}
\maketitle
\section{introdução}
\subsection{Biblioteca}
\begin{lstlisting}
#include<nome_da_biblioteca>
 Biblioteca e um conjunto de funcoes (pedacos de codigos) ja implementados
e que podem ser utilizados pelo programador
 Ex: stdio.h (entradas e saidas) ; stdlib.h (malloc, free); math.h .
Main
- Todo programa em C deve conter a funcao main().
- Return 0;
- Ex: int main(){}
\end{lstlisting}
\subsection{Operadores Aritméticos}
- Multiplicação: *
- Divisão inteira: int/int -> 7/2\\
- Divisão real: int/float -> 7/2.0\\
- subtração e adição padrão\\
Lembre-se precedência de operadores.\\
\subsection{Operadores Lógicos}
\begin{lstlisting}
- a && b
- a || b
- !a
I/O
Comando scanf(especificador, &variavel);
 Exemplo: int a;
scanf("%i", &a);
\end{lstlisting}
\subsection{comando printf}
\begin{lstlisting}
Comando printf (mensagem, variaveis)
 Exemplos: int a = 2;
 printf("%i\n", a);
 printf("Valor: %i\n", a);
\end{lstlisting}
\begin{lstlisting}
  int getchar(void): permite ler um unico caractere do teclado (le o codigo ASCII do
caracter)
#include <stdio.h>
#include<stdlib.h>
 int main() {
char c;
c = getchar();
printf("caractere: %c\n", c);
printf("codigo ASCII: %d\n", c);
system("pause");
return 0;
 }
\end{lstlisting}
\subsection{estrutura de seleção}
\begin{lstlisting}
Condicional simples
if (condicao){
expressoes;
expressoes;
}
Condicional if e else
if (condicao) {
comandos1;
 } else {
comandos2;
 }

 Condicional if e else
if (condicao1) {
comando1;
} else {
if (condicao2) {
comando2;
 } else {
comando4;
 }
} 
\end{lstlisting}

\end{document}